
%----------------------------------------------------------------------------------------
%	PREAMBUŁA
%----------------------------------------------------------------------------------------

\documentclass[12pt]{article}
\usepackage[polish]{babel}
\usepackage{polski}
\usepackage[utf8]{inputenc}
\usepackage{graphicx}
\usepackage{fancyhdr}
\usepackage{float}
\usepackage{graphicx}
\usepackage[hidelinks]{hyperref}
\usepackage{verbatim}
\usepackage{amsmath}
\usepackage{rotating}
\usepackage{listings}
\usepackage{xcolor}
\usepackage{subcaption}

\definecolor{lgray}{gray}{0.96}
\definecolor{lbcolor}{rgb}{0.9,0.9,0.9}
\lstset{
    framesep=2pt,
    breaklines=true,
    breakatwhitespace=true,
    basicstyle=\footnotesize,
    aboveskip={0.75\baselineskip},
    columns=flexible,
    showstringspaces=false,
    breaklines=true,
    prebreak = \raisebox{0ex}[0ex][0ex]{\ensuremath{\hookleftarrow}},
    frame=single,
    rulecolor=\color{lgray},
    showtabs=false,
    showspaces=false,
    showstringspaces=false,
    backgroundcolor=\color{lgray},
    identifierstyle=\ttfamily,
    keywordstyle=\color[rgb]{0,0,1},
    commentstyle=\color[rgb]{0.0,0.26,0.15},
    stringstyle=\color[rgb]{0.627,0.126,0.941}
}

\graphicspath{{static/}} 

\title{Algorytm Dijkstry}
\author{Arkadiusz Kasprzak, Aleksandra Poręba}

\makeatletter
\let\thetitle\@title
\let\theauthor\@author
\let\thedate\@date
\makeatother

%----------------------------------------------------------------------------------------
%	STRONA TYTUŁOWA
%----------------------------------------------------------------------------------------
\begin{document}
\begin{center}
\textsc{\normalsize Wydział Fizyki i Informatyki Stosowanej}\\[2.0cm] 
\includegraphics[scale = 1]{logo.pdf}\\[1cm] 
\textsc{\Large Systemy równoległe i rozproszone}\\[0.4cm] 


{ \huge \bfseries \LARGE{Drzewa wszystkich najkrótszych ścieżek - algorytm Dijkstry} }\\[0.2cm] 
{ \huge \bfseries \LARGE{Aplikacja PGAS} }\\[1cm] 

\flushright \Large Arkadiusz Kasprzak \\ Aleksandra Poręba

\vfill 

\center {\today}\\[2cm] 


\pagebreak 

\end{center}

%----------------------------------------------------------------------------------------
%	SPIS TREŚCI
%----------------------------------------------------------------------------------------
\setcounter{tocdepth}{2}
\tableofcontents
\pagebreak

%----------------------------------------------------------------------------------------
%	ZAWARTOŚĆ
%----------------------------------------------------------------------------------------

\pagestyle{fancy}
\fancyhf{}

\rhead{\theauthor}
\lhead{\thetitle}
\cfoot{\thepage}

\section{Wstęp}


\clearpage
\section{Obsługa programu}
Niniejsza część dokumentacji przedstawia informacje związane z kompilacją i uruchomieniem projektu.

\subsection{Obsługa projektu na pracowni WFiIS}

\subsection{Dane wejściowe}

\subsection{Format pliku wynikowego}


\newpage
\begin{thebibliography}{9}

\end{thebibliography}

\end{document}